%%%%%%%%%%%%%%%%%%%%%%%%%%%%%%%%%%%%%%%%%
% Short Sectioned Assignment LaTeX Template Version 1.0 (5/5/12)
% This template has been downloaded from: http://www.LaTeXTemplates.com
% Original author:  Frits Wenneker (http://www.howtotex.com)
% License: CC BY-NC-SA 3.0 (http://creativecommons.org/licenses/by-nc-sa/3.0/)
%%%%%%%%%%%%%%%%%%%%%%%%%%%%%%%%%%%%%%%%%

%----------------------------------------------------------------------------------------
%	PACKAGES AND OTHER DOCUMENT CONFIGURATIONS
%----------------------------------------------------------------------------------------

\documentclass[paper=a4, fontsize=11pt]{scrartcl} % A4 paper and 11pt font size

% ---- Entrada y salida de texto -----

\usepackage[T1]{fontenc} % Use 8-bit encoding that has 256 glyphs
\usepackage[utf8]{inputenc}
%\usepackage{fourier} % Use the Adobe Utopia font for the document - comment this line to return to the LaTeX default

% ---- Idioma --------

\usepackage[spanish, es-tabla]{babel} % Selecciona el español para palabras introducidas automáticamente, p.ej. "septiembre" en la fecha y especifica que se use la palabra Tabla en vez de Cuadro

% ---- Otros paquetes ----

\usepackage{url} % ,href} %para incluir URLs e hipervínculos dentro del texto (aunque hay que instalar href)
\usepackage{amsmath,amsfonts,amsthm} % Math packages
%\usepackage{graphics,graphicx, floatrow} %para incluir imágenes y notas en las imágenes
\usepackage{graphics,graphicx, float} %para incluir imágenes y colocarlas
% Para hacer cuadros
\usepackage{tcolorbox}
% Para hacer uso de párrafos
\usepackage{parskip}
\usepackage{hyperref}
\usepackage{vmargin}
\setpapersize{A4}
\setmargins{2.5cm}       % margen izquierdo
{1.5cm}                        % margen superior
{16.5cm}                      % anchura del texto
{23.42cm}                    % altura del texto
{10pt}                           % altura de los encabezados
{1cm}                           % espacio entre el texto y los encabezados
{0pt}                             % altura del pie de página
{2cm}                           % espacio entre el texto y el pie de página
% Para hacer tablas comlejas
%\usepackage{multirow}
%\usepackage{threeparttable}

%\usepackage{sectsty} % Allows customizing section commands
%\allsectionsfont{\centering \normalfont\scshape} % Make all sections centered, the default font and small caps

\usepackage{fancyhdr} % Custom headers and footers
\pagestyle{fancyplain} % Makes all pages in the document conform to the custom headers and footers
\fancyhead{} % No page header - if you want one, create it in the same way as the footers below
\fancyfoot[L]{Mario López González 3ºB} % Empty left footer
\fancyfoot[C]{} % Empty center footer
\fancyfoot[R]{\thepage} % Page numbering for right footer
\renewcommand{\headrulewidth}{0pt} % Remove header underlines
\renewcommand{\footrulewidth}{0pt} % Remove footer underlines
\setlength{\headheight}{13.6pt} % Customize the height of the header

\numberwithin{equation}{section} % Number equations within sections (i.e. 1.1, 1.2, 2.1, 2.2 instead of 1, 2, 3, 4)
\numberwithin{figure}{section} % Number figures within sections (i.e. 1.1, 1.2, 2.1, 2.2 instead of 1, 2, 3, 4)
\numberwithin{table}{section} % Number tables within sections (i.e. 1.1, 1.2, 2.1, 2.2 instead of 1, 2, 3, 4)

\setlength\parindent{0pt} % Removes all indentation from paragraphs - comment this line for an assignment with lots of text

\newcommand{\horrule}[1]{\rule{\linewidth}{#1}} % Create horizontal rule command with 1 argument of height


\title{	
\normalfont \normalsize 
\textsc{\textbf{Ingeniería de Servidores (2021-2022)} \\ Grado en Ingeniería Informática \\ Universidad de Granada} \\ [25pt] % Your university, school and/or department name(s)
\horrule{0.5pt} \\[0.4cm] % Thin top horizontal rule
\huge Memoria Práctica 4 \\ % The assignment title
\horrule{2pt} \\[0.5cm] % Thick bottom horizontal rule
}

\author{Mario López González} % Nombre y apellidos

\date{\normalsize\today} % Incluye la fecha actual

%----------------------------------------------------------------------------------------
% DOCUMENTO
%----------------------------------------------------------------------------------------

\begin{document}

\maketitle % Muestra el Título

\begin{figure}[H]
    \centering
    \includegraphics[scale=0.175]{images/server.jpg}
    \label{fig:server}
\end{figure}

\newpage %inserta un salto de página

\tableofcontents % para generar el índice de contenidos
\listoffigures % para generar el índice de imágenes

%%%%%%%%%%%%%%%%%%%%%%%%%%%%%%%%%%%%%%%%%%%%%%%%% 
% Apartado para la instalación en Ubuntu Server %
%%%%%%%%%%%%%%%%%%%%%%%%%%%%%%%%%%%%%%%%%%%%%%%%%
\newpage
\section{Phoronix}
\subsection{Instalación}
\subsubsection{Ubuntu Server}
Para instalar Phoronix en Ubuntu Server lo primero que tenemos que hacer es descargarnos la versión actual más estable con el siguiente comando:
\begin{tcolorbox}[colback=black!10, halign=left]
    \$ wget http://phoronix-test-suite.com/releases/repo/pts.debian/files/phoronix-test-suite\_8.6.0\_all.deb
\end{tcolorbox}

Una vez lo hemos descargado, procedemos a instalarlo con el comando dpkg:
\begin{tcolorbox}[colback=black!10, halign=left]
    \$ sudo dpkg -i phoronix-test-suite\_8.6.0\_all.deb
\end{tcolorbox}

Por último, solucionamos cualquier problema que pueda existir con las dependencias con el siguiente comando:
\begin{tcolorbox}[colback=black!10, halign=left]
    \$ sudo apt -f install
\end{tcolorbox}

\subsubsection{CentOS}
Para instalar Phoronix en CentOS lo primero que tenemos que hacer es descargarnos e instalar las dependencias necesarias para su ejecución:
\begin{tcolorbox}[colback=black!10, halign=left]
    \# sudo yum install wget php-cli php-xml bzip2
\end{tcolorbox}

Una vez hemos instalado las dependencias procedemos descargar la versión actual más estable para CentOS:
\begin{tcolorbox}[colback=black!10, halign=left]
    \# wget https://phoronix-test-suite.com/releases/phoronix-test-suite-8.6.0.tar.gz
\end{tcolorbox}

Cuando se hayan descargado los paquetes los extraemos:
\begin{tcolorbox}[colback=black!10, halign=left]
    \# tar xvfz phoronix-test-suite-8.4.1.tar.gz
\end{tcolorbox}

Por último, instalamos el fichero \textbf{\emph{install-sh}} que se encuentra dentro de la carpeta que acabamos de extraer:
\begin{tcolorbox}[colback=black!10, halign=left]
    \# sudo ./install-sh
\end{tcolorbox}

\subsection{Benchmarks}
\subsubsection{RAMspeed SMP}
Este benchmark se encarga de medir el rendimiento de los módulos RAM realizando una serie de operaciones que le indica el usuario. En mi caso,
he decidido ejecutar el test con la operación \emph{copy} con datos de tipo entero.

La ejecución de los test se hace con el comando "phoronix-test-suite run <benchmark>". En este caso la ejecución del benchmark sería de la siguiente manera:
\begin{tcolorbox}[colback=black!10, halign=left]
    \# phoronix-test-suite run ramspeed
\end{tcolorbox}

\begin{figure}[H]
    \centering
    \includegraphics[scale=0.36]{images/ramspeed_parametros.png}
    \caption{Parámetros para RAMspeed SMP}
    \label{fig:ramspeed_parametros}
\end{figure}

La imagen anterior contiene información sobre el hardware de nuestra máquina como puede ser la CPU o la cantidad de memoria principal instalada. Dicha información se muestra de
manera automática al ejecutar el test.

El test se ha ejecutado tanto en Ubuntu Server como en CentOS. En Ubuntu el tiempo estimado del test ha sido 22 minutos, mientras que en CentOS ha sido de 5 minutos. Por defecto, 
está programado para ejecutar tres test, aunque puede considerar conveniente realizar más test para una mayor precisión en sus cálculos. En mi caso, en CentOS ha realizado el
número de test predeterminados, sin embargo, en Ubuntu ha ocurrido lo contrario, en lugar de realizar tres test ha hecho seis. Debido al número de test realizados en Ubuntu cobra
sentido la diferencia de tiempo entre ambos test.

\begin{figure}[H]
    \centering
    \includegraphics[scale=0.5]{images/ramspeed_ubuntu.png}
    \caption{Ejecución de RAMspeed SMP en Ubuntu Server}
    \label{fig:ramspeed_ubuntu}
\end{figure}

\begin{figure}[H]
    \centering
    \includegraphics[scale=0.6]{images/ramspeed_centos.png}
    \caption{Ejecución de RAMspeed SMP en CentOS}
    \label{fig:ramspeed_centos}
\end{figure}

He calculado el coeficiente de variación ($cv$) de ambos test, $cv = (deviation / 100) / average$, concluyendo en que el test realizado en CentOS es más fiable que el realizado en Ubuntu.

La velocidad de los test deberían ser similares ya que se están ejecutando sobre el mismo hardware por lo que tomaré como velocidad de referencia la calculada en el test de CentOS.

\subsubsection{Sudokut}
Este benchmark mide el rendimiento de la CPU. Para ello, mide el tiempo que tarda el procesador en resolver 100 sudokus.

La ejecución de este test no arroja resultados tan dispares como en el test anterior. En este caso, la diferencia es minima, de unas pocas milésimas.

\begin{figure}[H]
    \centering
    \includegraphics[scale=0.5]{images/sudokut_ubuntu.png}
    \caption{Ejecución de Sudokut en Ubuntu Server}
    \label{fig:sudokut_ubuntu}
\end{figure}

\begin{figure}[H]
    \centering
    \includegraphics[scale=0.4]{images/sudokut_centos.png}
    \caption{Ejecución de Sudokut en CentOS}
    \label{fig:sudokut_centos}
\end{figure}

Como se puede observar en las imágenes anteriores, se han realizado el mismo número de tests y con la misma estimación de tiempo. Sin embargo, tras calcular los 
coeficientes de variación y su correspondiente comparativa, concluyo en que los resultados obtenidos en el test de Ubuntu son más fiables por lo que tomaré su
tiempo como valor de referencia.

\section{JMeter}
\subsection{Instalación de Docker}
Para este ejercicio es necesario instalar docker y docker-compose. A continuación mostraré los pasos para su instalación en Ubuntu Server:

Lo primero que tenemos que hacer es añadir la llave GPG para validar el repositorio de Docker que agregaremos posteriormente:
\begin{tcolorbox}[colback=black!10, halign=left]
    \$ sudo curl -fsSL https://download.docker.com/linux/ubuntu/gpg | sudo apt-key add -
\end{tcolorbox}

Acto seguido añadimos el repositorio con el siguiente comando (todo en la misma línea):
\begin{tcolorbox}[colback=black!10, halign=left]
    \$ sudo add-apt-repository "deb [arch=amd64] https://download.docker.com/linux/ubuntu \$(lsb\_release -cs) stable"
\end{tcolorbox}

Una vez añadido el repositorio procedemos a actualizar la lista de repos y a instalar docker-ce (community edition):
\begin{tcolorbox}[colback=black!10, halign=left]
    \$ sudo apt update \\
    \$ sudo apt install docker-ce
\end{tcolorbox}

Por último, instalamos docker-compose y comprobamos que se ha instalado:
\begin{tcolorbox}[colback=black!10, halign=left]
    \$ sudo apt install docker-compose \\
    \$ docker-compose --version
\end{tcolorbox}

\subsection{Instalación de iseP4JMeter}
Descargamos la aplicación del repositorio proporcionado:
\begin{tcolorbox}[colback=black!10, halign=left]
    \$ git clone https://github.com/davidPalomar-ugr/iseP4JMeter.git
\end{tcolorbox}

Una vez se ha descargado accedemos a la carpeta y levantamos la imagen de docker:
\begin{tcolorbox}[colback=black!10, halign=left]
    \$ sudo docker-compose up
\end{tcolorbox}
\begin{figure}[H]
    \centering
    \includegraphics[scale=0.4]{images/docker_compose.png}
    \caption{docker-compose up}
    \label{fig:docker_compose}
\end{figure}

Por último, detenemos el contenedor y habilitamos el puerto 3000 ya que en el repositorio especifica que es el puerto usado por la aplicación y el que utiliza la base
de datos MongoDB:
\begin{tcolorbox}[colback=black!10, halign=left]
    \$ sudo ufw allow 3000/tcp \\
    \$ sudo ufw allow 27017/tcp
\end{tcolorbox}

Volvemos a levantar el servicio e ingresamos en la siguiente dirección para comprobar que la aplicación funciona:
\begin{tcolorbox}[colback=black!10, halign=left]
    192.168.56.110:3000
\end{tcolorbox}
\begin{figure}[H]
    \centering
    \includegraphics[scale=0.33]{images/iseP4JMeter.png}
    \caption{Aplicación web}
    \label{fig:iseP4JMeter}
\end{figure}

\subsection{Instalación de JMeter (MacOS)}
En mi caso instalo JMeter en mi sistema nativo, MacOS, haciendo uso del gestor de paquetes \emph{brew}:
\begin{tcolorbox}[colback=black!10, halign=left]
    \$ brew install jmeter
\end{tcolorbox}

\begin{figure}[H]
    \centering
    \includegraphics[scale=0.33]{images/jmeter.png}
    \caption{JMeter}
    \label{fig:jmeter}
\end{figure}

\subsection{Plan de JMeter}
Crearemos un test que simule las peticiones al servidor por parte de los usuarios que accedan a la web. Para crear dicha prueba cambiaremos el nombre del plan que aparece
por defecto a "ETSIIT Alumnos API". En él definiremos dos variables para parametrizar la IP del servidor y el puerto:
\begin{itemize}
    \item \textbf{host:} 192.168.56.105
    \item \textbf{port:} 3000
\end{itemize}

\begin{figure}[H]
    \centering
    \includegraphics[scale=0.45]{images/variables.png}
    \caption{Parametrización de variables}
    \label{fig:variables}
\end{figure}

\subsubsection{HTTP Authorization Manager - Basic-Auth Api}
Este elemento, creado dentro del plan, contiene la URL de la aplicación web junto con sus credenciales necesarias para el acceso ya que está protegido. Por lo tanto,
crearemos una autorización para la dirección de la API con su respectivo usuario y contraseña:
\begin{itemize}
    \item \textbf{Base URL:} http://\${host}:\${port}/api/v1/auth/login
    \item \textbf{Username:} etsiiApi
    \item \textbf{Password:} laApiDeLaETSIIDaLache
\end{itemize}

\begin{figure}[H]
    \centering
    \includegraphics[scale=0.4]{images/autorizacion.png}
    \caption{Autorización}
    \label{fig:autorizacion}
\end{figure}

\subsubsection{Grupo de hebras - Alumno}
Crearemos un grupo de hebras dentro del plan de tal forma que estas que actuarán como alumnos. La configuramos de la siguiente manera:
\begin{figure}[H]
    \centering
    \includegraphics[scale=0.35]{images/hebras_alumnos.png}
    \caption{Grupo de hebras de alumnos}
    \label{fig:hebras_alumnos}
\end{figure}

\subsubsection{CSV Data Set Config - Credenciales Alumno}
Para automatizar el proceso de ''usuarios'' $ $ que acceden a la página web utilizaremos el archivo ''alumnos.csv'' $ $ que contiene los nombres de usuario y contraseñas
que tienen acceso a la web. Para ello, crearemos un elemento ''CSV Data Set Config'' $ $ que configuraremos de la siguiente forma:
\begin{figure}[H]
    \centering
    \includegraphics[scale=0.33]{images/csv_alumnos.png}
    \caption{Datos CSV de alumnos}
    \label{fig:csv_alumnos}
\end{figure}

\subsubsection{HTTP Request - Login Alumno}
Crearemos una petición HTTP para que las hebras del grupo soliciten acceso como usuarios, para ello creamos un elemento "HTTP Request" $ $ dentro del grupo de hebras:
\begin{figure}[H]
    \centering
    \includegraphics[scale=0.33]{images/login_alumnos.png}
    \caption{Petición HTTP Alumnos}
    \label{fig:login_alumnos}
\end{figure}

\subsubsection{Regular Expression Extractor - Extract JWT Token}
Añadiremos un extractor de expresiones regulares para guardar las credenciales de alumno usadas para la petición HTTP en la variable \emph{token}:
\begin{figure}[H]
    \centering
    \includegraphics[scale=0.6]{images/token_alumnos.png}
    \caption{Extractor de Tokens para alumnos}
    \label{fig:token_alumnos}
\end{figure}

\subsubsection{Espera aleatoria}
Para simular las gestiones del usuario añadiremos una espera aleatoria con el elemento "Gaussian Random Timer".
\begin{figure}[H]
    \centering
    \includegraphics[scale=0.6]{images/timer.png}
    \caption{Temporizador aleatorio gaussiano}
    \label{fig:timer}
\end{figure}

\subsubsection{HTTP Request - Recuperar Datos Alumno}
Añadiremos una nueva petición HTTP para consultar los datos de acceso del alumno:
\begin{figure}[H]
    \centering
    \includegraphics[scale=0.33]{images/recuperar_alumnos.png}
    \caption{Petición HTTP para recuperar los datos de los alumnos}
    \label{fig:recuperar_alumnos}
\end{figure}

\subsubsection{HTTP Header Manager - JWT Token}
Para que las cabeceras de las peticiones HTTP contengan las credenciales de los alumnos debemos añadir un elemento ''HTTP Header Manager'': 
\begin{figure}[H]
    \centering  
    \includegraphics[scale=0.33]{images/header_alumnos.png}
    \caption{Gestor de cabeceras HTTP para alumnos}
    \label{fig:header_alumnos}
\end{figure}

\subsubsection{Grupo de hebras - Administrador}
Para simular el acceso de administradores añadiremos un nuevo grupo de hebras dentro del plan, al nivel del grupo de hebras de alumnos, de la siguiente forma
que añadimos el de alumnos.

\subsubsection{CSV Data Set Config - Credenciales Administrador}
Para leer las credenciales de los administradores crearemos un elemento ''CSV Data Set Config'' idéntico al usado en ''Credenciales Alumno'' con la diferencia de 
que el archivo usado esta vez es ''administradores.csv''.

\subsubsection{HTTP Request - Login Administrador}
Del mismo modo que con ''Login Alumno'', creamos una petición HTTP para las hebras soliciten acceso como administradores. 

\subsubsection{Regular Expression Extractor - Extract JWT Token}
Añadiremos un extractor de expresiones regulares para guardar las credenciales del administrador usadas para la petición HTTP en la variable \emph{token}:

\subsubsection{Access Log Sampler - Accesos Administradores}
Añadiremos un elemento ''Access Log Sampler'' para realizar un muestreo y simular el acceso de los administradores.
\begin{figure}[H]
    \centering  
    \includegraphics[scale=0.33]{images/access_administrador.png}
    \caption{Access Log Sampler para administradores}
    \label{fig:access_administrador}
\end{figure}

\subsubsection{HTTP Header Manager - JWT Token}
Para que las cabeceras de las peticiones HTTP contengan las credenciales de los administradores debemos añadir un elemento ''HTTP Header Manager'' tal y como hemos hecho para el
grupo de hebras de alumnos.

\subsubsection{Espera aleatoria}
Para simular las gestiones del usuario añadiremos una espera aleatoria con el elemento "Gaussian Random Timer".

\newpage

\begin{thebibliography}{0}
    \bibitem{} \href{}{}
\end{thebibliography}

\end{document}
